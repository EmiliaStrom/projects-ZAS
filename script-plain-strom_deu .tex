\documentclass[11pt, oneside]{article}   	% use "amsart" instead of "article" for AMSLaTeX format
\usepackage{geometry}                		% See geometry.pdf to learn the layout options. There are lots.
\geometry{a4paper}                   		% ... or a4paper or a5paper or ... 
%\geometry{landscape}                		% Activate for for rotated page geometry
%\usepackage[parfill]{parskip}    		% Activate to begin paragraphs with an empty line rather than an indent
\usepackage{graphicx}				% Use pdf, png, jpg, or eps? with pdflatex; use eps in DVI mode
								% TeX will automatically convert eps --> pdf in pdflatex		
\usepackage{amssymb}
\usepackage{xcolor}

\title{Cum Sine -- Skript f\"ur Experimenter}
\author{}
\date{}							% Activate to display a given date or no date

\begin{document}
\maketitle

%%%%%%%%%%%%%%%%%%%%%%%%
%PRELIMINARY CONTENT
%%%%%%%%%%%%%%%%%%%%%%%%
\section{Anmerkungen}

\begin{itemize}
\item{E: Experimenter} 
\item{T: Teilnehmer am Experiment}
\end{itemize}


% -----------------------------------------------------------------------------------------------------------------------------------------------------------------

\section{Vorbereitung}
{\it Anleitung f\"ur die Eltern}
\begin{itemize}
\item  Den Eltern soll am Anfang kurz erkl\"art werden, wie das Experiment abl\"auft.
\item  Den Eltern muss vorher gesagt werden, dass sie die Kinder auf keinen Fall korrigieren sollten oder sonstige positive oder negative R\"uckmeldung geben sollten. Weiter soll erkl\"art werden, dass die starke Bindung zwischen Elternteil und Kind dazu f\"uhrt, dass sogar kaum wahrnehmbare zustimmende oder ablehnende Zeichen/Gesten/Gesichtsausdr\"ucke des Elternteils das Verhalten des Kindes beeinflussen k\"onnen. 
\item Die Eltern k\"onnen jedoch ihre Kinder  ermutigen, etwas zu sagen  -- also einfach die Antwort zu geben, die das Kind geben m\"ochte (ohne Reaktion auf die Antwort selbst, wie oben beschrieben).
\begin{itemize}
\item 
Statt einer Reaktion des Elternteils gibt es im Experiment selbst ein Signal und eine standardisierte ermutigende Reaktion des Experimenters. 
\end{itemize}
\item Man sollte die Eltern auch informieren,  dass die Kinder im Experiment bestimmte Ausdr\"ucke oder S\"atze benutzen werden, die f\"ur Erwachsene vielleicht seltsam oder falsch klingen. Dies ist ein normaler Teil der Sprachentwicklung von jedem Kind, und gibt keinen Aufschluss \"uber deren Kompetenz oder Entwicklung. 
\item Eltern werden darauf hingewiesen, dass sie das Informationsblatt zum Datenschutz downloaden können.
\end{itemize}

% -----------------------------------------------------------------------------------------------------------------------------------------------------------------


\section{\"Uberpr\"ufen von Ton und Lautst\"arke }

\begin{itemize}
\item Eltern sollen das `play' Symbol des ersten Bildschirms anklicken, um den Ton zu testen. 
\item Wenn das Signal nicht h\"orbar ist, oder zu laut ist, sollten die Eltern die Lautst\"arke anpassen. 
\item Jede Interaktion mit dem Kind endet mit einem `ermutigendem' Signal. 
\item Das Kind soll aber keine Angst haben oder das Signal als unangenehm empfinden. 
\end{itemize}

% -----------------------------------------------------------------------------------------------------------------------------------------------------------------

\section{Einleitung} 

{\it Slide: Herzlich willkommen, gleich geht es los!}

\begin{itemize}
\item [E:] Hallo! Hier k\"{o}nnen Sie bitte auf das ``Play''-Symbol klicken, dann wird ein Ton abgespielt. Bitte überprüfen Sie, ob die Lautstärke so OK ist. Ansonsten passen Sie die Lautstärke bitte an.
\item [E:] Auf dem n\"achsten Bildschirm beginnt unser Spiel, das ich  vor allem mit [Name des Kindes] zusammen spielen  m\"ochte. 
\end{itemize}

\noindent {\it E tippt `start' (ohne Anf\"uhrungszeichen) ein, um sich als Experimenter einzuloggen. }

% -----------------------------------------------------------------------------------------------------------------------------------------------------------------


\section{Zustimmung des Kindes}

\begin{itemize}
\item [E:]  Hallo, ich m\"ochte gerne ein Spiel spielen. In dem Spiel gibt es viele Tiere. M\"ochtest du mitspielen?
\item [T:]  Ja!
\item [E:]  Super! Wenn du mitmachen m\"ochtest, dr\"ucke bitte auf das l\"achelnde Gesicht.
\end{itemize}

% -----------------------------------------------------------------------------------------------------------------------------------------------------------------

\section{Einf\"uhrung von Ablauf und Hand-Element}

\begin{itemize}
\item [E:] Jetzt erkl\"are ich dir das Spiel.
\item [E:] In dem Spiel schauen wir uns Tiere an.
\item [E:] Was f\"ur ein Tier ist das? ({\it Bild: Katze})
\item [T:] Eine Katze.
\item [E:] Genau!! Eine Katze.
\end{itemize}

{\it Falls das Kind nicht `Katze' sagt:}

\begin{itemize}
\item [E:] Das ist eine Katze. Richtig?
\end{itemize}

\noindent {\it N\"achster Bildschirm: Die Hand erscheint}

\begin{itemize}
\item [E:] Kannst du die Hand sehen?
\item [T:] Ja!
\item [E:] Die Hand zeigt auf die Katze.
\item [E:] Ich werde dich in diesem Spiel immer fragen, auf welches Tier die Hand zeigt. Kannst du mir dann immer sagen, auf welches Tier die Hand zeigt?
\item [T:] Ja!
\end{itemize}

{\it N\"achster Bildschirm}

\section{Eingew\"ohnungsphase f\"ur die Tiere}
\begin{itemize}
\item Es erscheinen immer 2 Tiere
\item Die Hand zeigt immer auf eines der Tiere
\end{itemize}


\begin{itemize}
\item[] {\it Hand erscheint}
\item [E:] Welches Tier ist das?
\item [T:] Die Katze.
\item [E:] Prima!/Super!
\item[] {\it Hand erscheint}
\item [E:] Welches Tier ist das?
\item [T:] Der Hund.
\item [E:] Super!/Prima!/
\item[] {\it Hand erscheint}
\end{itemize}

Wenn das Kind nicht den richtigen Tiernamen sagt, sage bitte:
\begin{itemize}
\item [E:] Das ist eine Katze, oder?
\end{itemize}
\item [] \textit{Es soll sichergestellt werden, dass die Kinder die Namen der Tiere in dem Experiment kennen.}

\item [] \textit{Und so weiter f\"ur alle anderen Tiere (noch zwei Runden)}

% -----------------------------------------------------------------------------------------------------------------------------------------------------------------

\section{Eingew\"ohnung f\"ur den Hauptteil des Experiments}

\begin{itemize}
\item Das Ziel ist, dass die Kinder die Tiere beschreiben, indem sie die Kleidungsst\"ucke erw\"ahnen 
\end{itemize}

\begin{itemize}
\item [E:] Schau mal! Da sind zwei Schweine!
\end{itemize}

\noindent {\it Als n\"achstes erscheint die Hand}

\begin{itemize}
\item [E:] Welches Schwein ist das?
\end{itemize}

\noindent {\it Wenn das Kind nichts sagt, sagt E:}

\begin{itemize}
\item[E:] Ist das eine Schwein anders als das andere?/Gibt es etwas, was anders ist bei einem von den Schweinen?/
\item[E:]  Ist da was besonderes an dem Schwein auf das die Hand zeigt?
\end {itemize}

\noindent {\it N\"achster Bildschirm: Die  Hand zeigt auf das andere Schwein}

\begin{itemize}
\item [E:] Welches Schein ist das?
\item [T:] Das Schwein mit Schuhen.
\item [E:] Super!
\end{itemize}

\noindent {\it Es gibt ein zweites Paar, mit denselben Tieren. Die selbe Prozedur wird dann mit diesen wiederholt. }

% -----------------------------------------------------------------------------------------------------------------------------------------------------------------

\section{Priming ``ohne''}
\begin{itemize}
\item [E:] Schau mal, hier sind zwei Bären. Einer mit Mütze und einer ohne Mütze.
\item [] \textit{Hand erscheint}
\item [E:] Welcher Bär ist das?
\item [T:]Der Bär ohne Mütze.
\item [E:] Genau, super!
\item [] \textit{Hand erscheint}
\item [E:] Welcher Bär ist das?
\item [T:] Der Bär mit Mütze.
\item [E:] Genau, super!
\end{itemize}

% -----------------------------------------------------------------------------------------------------------------------------------------------------------------

\section{Hauptteil des Experiments}

\begin{itemize}
\item Das Kind wird 6 target items und 6 fillers sehen, immer abwechselnd
\item Wenn die Aufzeichnung l\"auft, werden sowohl E als auch T aufgenommen 
\item Die Aufzeichnung kann erst nach 10 Sekunden gestoppt werden
\end{itemize}

\noindent  {\it Die Tiere erscheinen}\\
{\it Die Hand erscheint und die Aufzeichnung beginnt} 

\vspace{1em}
\begin{itemize}
\item [E:] Welcher B\"{a}r ist das?
\item [P:] Der B\"ar ohne M\"utze.
\item [E:] Prima!/Super!
\end{itemize}

\noindent {\it N\"achster Bildschirm}

\noindent {\it Und so weiter}



\end{document}